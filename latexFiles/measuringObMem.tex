As a first step towards understanding memorability of objects, we built an image database containing a variety of objects from a diverse range of categories, and measured the probability that every object in each image will be remembered by a large group of subjects after a single viewing. This helps provide ground truth memorability scores for the objects inside the images and allows for a precise analysis of the memorable elements within an image. For this task, we utilized the PASCAL-S dataset \cite{yin14}, a fully segmented dataset built on the validation set of the PASCAL VOC 2010 \cite{pascal10} segmentation challenge. For improved segmentation purposes, we manually cleaned up and refined the segmentations from this dataset. We removed all homogenous non-object or background segments such as ground, grass, floor, sky etc, as well as imperceptible object fragments and excessively blurred regions. All remaining object segmentations were tested for memorability. In the end, our final dataset consisted of $850$ images and $3412$ object segmentations i.e. on average each image consisted of approximately $4$ object segments for which we gathered the ground truth memorability on.

\subsection{Memory Game: Measuring Object Memorability}
To measure the memorability of individual objects from our dataset, we created an alternate version of the Visual Memory Game through Amazon Mechanical Turk following the basic design in \cite{isola11}, with the exception of a few key differences. In our game, participants first viewed a sequence of images one at a time, with a $1.5$ second gap in between image presentations. Subjects were asked to remember the contents and objects inside those images as much as they could. To ensure that subjects would not just only look at the salient or center objects, subjects had unlimited time to freely view the images. Once they were done viewing an image, they could press any key to advance to the next image. Following the initial image sequence, participants then viewed a sequence of objects, their task then being to indicate through a key press which of those objects was present in one of the previously shown images. Each object was displayed for $1.5$ second, with a $1.5$ second gap in between the object sequences. Pairs of corresponding image and object sequences were broken up into $10$ blocks. Each block consisted of $80$ total stimuli ($35$ images and $45$ objects), and lasted approximately $3$ minutes. At the end of each block, the subject could take a short break. Overall, the experiment took approximately took $30$ minutes to complete.

Unknown to the subjects, inside each block, each sequence of images was pseudo-random and consisted of $3$ 'target' images taken from the Pascal-S dataset whose objects the participants were to later identify. The remaining images in the sequence consisted of $16$ 'filler' images and $16$ 'familiar' images. The 'filler' images were randomly selected from the DUT-OMRON dataset \cite{dutomron13} and the 'familiar' images were randomly sampled from the MSRA dataset proposed in \cite{msra11}. Similarly, the object sequence was also pseudo-random and consisted of $3$ 'target' objects ($1$ object taken randomly from each previously shown target image). The remaining objects in the sequence consisted of $10$ 'control' objects, $16$ 'filler' objects, and $16$ 'familiar' objects. The 'filler' objects were taken randomly from the $80$ different object categories in the Microsoft COCO dataset \cite{coco14} and the 'familiar' objects were the objects taken from the previously displayed 'familiar' images in the image sequence. The fillers and familiars helped provide spacing between the target images and target objects, whereas the control objects allowed us to check if the subjects were paying attention to the task \cite{brady08}, \cite{isola11}. While the fillers and familiars (both the images and objects) were taken from datasets resembling real world scenes and objects, the 'control' objects were artificial stimuli randomly sampled from the dataset proposed in \cite{brady08} and helped serve as a control to test the attentiveness of the subjects. The target images and the respective target objects were spaced $70-79$ stimuli apart, and familiar images and their respective objects were spaced $1-79$ stimuli apart. All images and objects appeared only once, and each subject was tested on only one object from each target image. Objects were centered within their parent frame and non-object pixels were set to grey. Participants were required to complete the entire task, which included $10$ blocks (overall time approximately $30$ minutes), and could not participate in the experiment a second time. After collecting the data, we assigned a 'memorability score' to each target object in our dataset, defined as the percentage of correct detections by subjects. In all our analysis, we removed all subjects whose accuracy on the control objects was below $70\%$. In the end, our analysis was performed on a total of {$1823$} workers from Mechanical Turk ($> 95\%$ approval rate in Amazon’s system).  The memorability score of an object corresponded to the number of subjects that correctly detected the repetition of that object. On average, each object was scored by $16$ subjects and the average memorability score was $33\%$ ($SD = 28\%$). 

\subsection{Consistency Analysis}
Previous work on image memorability has found human consistency to be fairly high. That is, people tend to remember the same images, and exhibit similar performance in doing so. Despite variability due to individual differences and other sources of noise, this level of consistency provides evidence that memorability is an intrinsic property of images that can be predicted. In contrast to full images, this paper focuses primarily on the memorability of individual objects in an image, which may or may not exhibit the same level of human consistency as full images, which often contain complex arrangements of several objects. High consistency in object memorability would indicate that, like full images, objects can potentially be predicted with high accuracy. To assess human consistency in remembering objects, we repeatedly divided our entire subject pool into two equal halves and quantified the degree to which memorability scores for the two sets of subjects were in agreement using Spearman’s rank correlation (ρ). We computed the average correlation over 25 of these random split iterations, yielding a final value of 0.76. Such a result confirms that human consistency in remembering particular objects is at least as strong as that of images.   