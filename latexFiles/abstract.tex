Recent work by Isola et. al. (2011) has demonstrated that memorability is an intrinsic property of images that is consistent across viewers and can be predicted accurately with current computer vision techniques. Despite progress, a clear understanding of the specific components of an image that drive memorability are still unknown. While previous studies such as Khosla et. al. (2012) have tried to investigate computationally the memorability of image regions within individual images, no behavioral study has systematically explored which memorability of image regions. Here we study which region from an image is memorable or forgettable. Using a large image database, we obtained the memorability scores of the different visual regions present in every image. In our task, participants viewed a series of images, each of which were displayed for 1.4 seconds. After the sequence was complete, participants similarly viewed a series of image regions and were asked to indicate whether each region was seen in the earlier sequence of full images. 